\documentclass[10pt]{beamer}

\usetheme[progressbar=frametitle]{metropolis}

\usepackage[T1]{fontenc}
\usepackage{newunicodechar}
%\usepackage[utf8]{inputenc}

\usepackage{subcaption}
\usepackage{adjustbox}
\usepackage{booktabs}
\usepackage[scale=2]{ccicons}

% For pseudo codes
\usepackage{algorithm}
\usepackage[noend]{algpseudocode}
\makeatletter
\def\BState{\State\hskip-\ALG@thistlm}
\makeatother
%

\usepackage{multirow}
\usepackage[none]{hyphenat}
\usepackage{textcomp}
\usepackage{gensymb}
\sloppy 
%\usebackgroundtemplate


\usepackage{pgfplots}
\usepgfplotslibrary{dateplot}

\usepackage{xspace}
\newcommand{\themename}{\textbf{\textsc{metropolis}}\xspace}

\setbeamercolor{background canvas}{bg=white!20}

\title{Title}
\subtitle{Sub-title (Optional)}
\date{Jun 16th - 20th, 2019}
\author{Author 1 and Author 2}
\institute{Affiliation - Country}
\titlegraphic{\small\center International Conference on Power Systems Transients
\\Perpignan, France\\June 16th – 20th, 2019

\vspace{-15mm}\flushright\includegraphics[height=1.50cm]{logoIPST2019}}



% logo of IPST 2019
\logo{\includegraphics[width=1cm]{logoIPST2019}\hfill}
\newcommand{\nologo}{\setbeamertemplate{logo}{}} % command to set the logo to nothing
\newcommand{\congress}{IPST 2019 June 16th - 20th, Perpignan, France.}

% footer
\makeatletter
\setbeamertemplate{footline}
{
  \leavevmode%
  \hbox{%

  \begin{beamercolorbox}[wd=.9\paperwidth,ht=2.25ex,dp=1ex,center]{institute in head/foot}%
    \usebeamerfont{abstract}%
    \congress
  \end{beamercolorbox}%

  \begin{beamercolorbox}[wd=.1\paperwidth,ht=2.25ex,dp=1ex,right]{institute in head/foot}%
    \usebeamerfont{abstract} 
    \insertframenumber{} / \inserttotalframenumber\hspace*{2ex} 
  \end{beamercolorbox}}%
  
}
\makeatother




% Document begin %%%%%%%%%%%%%%%%%%%%%%%%%%%%%%%%%%%%%%%%%%%%%%%%%%%%%%%%%%%%%%%%%%%%
\begin{document}

\maketitle
%%%%%%%%%%%%%%%%%%%%%%%%%%%%%%%%%%%%%%%%%%%%%%%%%%%%%%%%%%%%%%
\section{Introduction} 



%%%%%%%%%%%%%%%%%%%%%%
{
\begin{frame}[fragile]{Including a simple animation}
\only<1>{
    \begin{figure}[ht]
      \centering{}\includegraphics[trim = 0mm 0mm 0mm 0mm, clip, width=0.5\linewidth]{logoIPST2019}%\caption{Research Gant diagram.\label{fig:Schedule}}
    \end{figure}
}
\only<2>{
    \begin{figure}[ht]
      \centering{}\includegraphics[trim = 0mm 0mm 0mm 0mm, clip, width=0.6\linewidth]{logoIPST2019}%\caption{Research Gant diagram.\label{fig:Schedule}}
    \end{figure}
}
\end{frame}
}

%%%%%%%%%%%%%%%%%%%%%%
\begin{frame}[fragile]{Motivation}
Including an image

    \begin{figure}[ht]
      \hspace*{-1cm}\includegraphics[width=0.5\linewidth]{logoIPST2019}
    \end{figure}
\end{frame}
%%%%%%%%%%%%%%%%%%%%%%
\begin{frame}[fragile]{Including items}
    \begin{itemize}
        \item Item name
\pause        
        \item Item name
    \end{itemize}
\end{frame}
%%%%%%%%%%%%%%%%%%%%%%
\begin{frame}[fragile]{General Objective}
    
    Normal text
    
\end{frame}

%%%%%%%%%%%%%%%%%%%%%%%%%%%%%%%%%%%%%%%%%%%%%%%%%%


%%%%%%%%%%%%%%%%%%%%%%%%%%%%%%%%%%%%%%%%%%
\section{Another section}
%%%%%%%%%%%%%%%%%%%%%%
\begin{frame}[fragile]{Equations}

Examples:

    \begin{equation}\label{eq:funtcionName}
        f_1 = 60
    \end{equation}

Some text
\pause
\textcolor{red}{To remark something}
\end{frame}
%%%%%%%%%%%%%%%%%%%%%%

%%%%%%%%%%%%%%%%%%%%%%%%%%%%%%%%%%%%%%%%%%%%%%%%%%%%%%%%%%%%%%
\section{Another section}
%%%%%%%%%%%%%%%%%%%%%%

%%%%%%%%%%%%%%%%%%%%%%
\begin{frame}[fragile]{Including pseudo-code}

\begin{algorithm}[H]
    \caption{Algorithm}\label{algorithmLabel}
    \begin{algorithmic}[1]
        \State Initialize
        \State $\mathbf{a} \gets N$ do something
        \While {$Condition$}
            \State Do something
            \If {$Condition$}
                \State do something
            \EndIf
        \EndWhile
    \end{algorithmic}
\end{algorithm}

\end{frame}

\section{Conclusions}
\begin{frame}[fragile]{Conclusions}

    \begin{itemize}
        \item Conclusion 1 \cite{Murata}
        \pause
        \item Conclusion 2
    \end{itemize}
\end{frame}


%%%%%%%%%%%%%%%%%%%%%%%%%%%%%%%%%%%%%%%%%%%%%%%%%%%%%%%%%%%%%%
{\nologo
\begin{frame}[allowframebreaks]
        \frametitle{References}
        \bibliographystyle{abntex2-alf}
        \bibliography{ref.bib}
\end{frame}
%%%%%%%%%%%%%%%%%%%%%%
%\newcommand{\nologo}{\setbeamertemplate{logo}{}} % command to set the logo to nothing

\begin{frame}[standout]
  Thank you \\ Questions?
\end{frame}
}
\end{document}
